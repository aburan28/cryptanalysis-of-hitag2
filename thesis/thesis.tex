\documentclass[12pt]{report}

\usepackage{hyperref}
\usepackage[all]{hypcap}
\usepackage{listings}
\usepackage{appendix}
\usepackage{amsmath, amsthm, amssymb}
\usepackage{amsfonts}
\usepackage{graphicx}
\usepackage{placeins}
\usepackage{url}
\usepackage{color}
\usepackage{fancyhdr}
\usepackage{rotating}
\usepackage{array}


\hypersetup{
%    bookmarks=true,         % show bookmarks bar?
    unicode=false,          % non-Latin characters in Acrobat�s bookmarks
    pdftoolbar=true,        % show Acrobat�s toolbar?
    pdfmenubar=true,        % show Acrobat�s menu?
    pdffitwindow=false,     % page fit to window when opened
    pdfnewwindow=true,      % links in new window
    colorlinks=false,       % false: boxed links; true: colored links
    linkcolor=red,          % color of internal linksline-numbers
    citecolor=green,        % color of links to bibliography
    filecolor=magenta,      % color of file links
    urlcolor=cyan           % color of external links
}

\lstset{ 
language=C,                			% choose the language of the code
basicstyle=\footnotesize,       % the size of the fonts that are used for the code
showstringspaces=false,         % underline spaces within strings
numbers=none,                   % where to put the line-numbers
numberstyle=\footnotesize,      % the size of the fonts that are used for the line-numbers
stepnumber=1,                   % the step between two . If it's 1 each line will be numbered
numbersep=5pt,                  % how far the line-numbers are from the code
backgroundcolor=\color{white},  % choose the background color. You must add \usepackage{color}
showspaces=false,               % show spaces within strings adding particular underscores
showtabs=false,                 % show tabs within strings adding particular underscores
escapeinside={\%*}{*)}          % if you want to add a comment within your code
}
%% Define a new 'leo' style for the package that will use a smaller font.
\makeatletter
\def
\url@leostyle{%
  \@ifundefined{selectfont}{\def\UrlFont{\sf}}{\def\UrlFont{\small\ttfamily}}}

\makeatother
%% Now actually use the newly defined style.
\urlstyle{leostyle}

\begin{document}

\begin{titlepage}

\thispagestyle{empty}
\rule{0pt}{50pt}

\begin{center}
 	\huge{Time-memory-data \\cryptanalysis of \\the HiTag2 stream cipher}\\
 		\vspace{2.5cm}
 	\Large{Master thesis project}\\
 	\Large{\today}\\
\end{center}

\vspace{5cm}

\begin{center}
	\Large{Rajesh Bachani \\(s061332)}\\
\end{center}
\end{titlepage}

\newpage
\thispagestyle{empty}
\vspace*{11cm}
{\noindent Department of Mathematics}\\
{Technical University of Denmark}\\
{Matematiktorvet, Building 303}\\
{DK-2800 Kgs.~Lyngby, Denmark}\\
{Phone +45 45253031, Fax +45 45881399}\\
{instadm@mat.dtu.dk}\\
{www.mat.dtu.dk}

\newpage
\pagenumbering{roman}
\setcounter{page}{1}
\pagestyle{fancy}
\addcontentsline{toc}{chapter}{Preface}

\chapter*{Preface}

This thesis was prepared at the Department of Mathematics, the Technical University of Denmark in partial fulfillment of the requirements for acquiring the Master of Science degree in Computer Science and Engineering.

The thesis is based on the implementation and evaluation of tradeoff attacks on the HiTag2 stream cipher. Three important tradeoff parameters are considered including the time taken for the attack, memory required for precomputation and the length of output available from the stream cipher. The thesis presents cryptanalysis results for four different tradeoff attacks by varying these parameters for each attack. 

\vspace{20mm}
\mbox{}\hfill
\begin{minipage}[t]{80mm}
Lyngby, September 2008\\
 \\
Rajesh Bachani
\end{minipage}

\addcontentsline{toc}{chapter}{Acknowledgements}

\chapter*{Acknowledgements}

I would like to express my deep-felt gratitude to my advisor, Dr. Vladik 
Kreinovich of the Computer Science Department at The University of Texas at El
Paso, for his advice, encouragement, enduring patience and constant support.
He was never ceasing in his belief in me (though I was often doubting in my 
own abilities), always providing clear explanations when I was (hopelessly) 
lost, constantly driving me with energy ({\it Where does he get it?!\/}) when 
I was tired, and always, {\em always\/} giving me his time, in spite of 
anything else that was going on.  His response to my verbal thanks one day was 
a very modest, ``It's my job.''  I wish all students the honor and opportunity 
to experience his ability to perform at that job.

I also wish to thank the other members of my committee, Dr. Luc Longpr\'{e} of 
the Computer Science Department and Dr. Mohamed Amine Khamsi of the Mathematics 
Department, both at The University of Texas at El Paso.  Their suggestions, 
comments and additional guidance were invaluable to the completion of this work.
As a special note, Dr. Longpr\'{e} graciously volunteered to act as my advisor
while Dr. Kreinovich was working abroad in Europe.  He was extremely helpful in 
providing the additional guidance and expertise I needed in order to complete 
this work, especially with regard to the chapter on NP-hard problems and the 
theory of NP-completeness.

\newpage
Additionally, I want to thank The University of Texas at El Paso Computer 
Science Department professors and staff for all their hard work and dedication,
providing me the means to complete my degree and prepare for a career as a 
computer scientist. This includes (but certainly is not limited to) the 
following individuals:

\bigskip

\noindent
Dr. Andrew Bernat
\begin{quote}
  He made it possible for me to have many wonderful experiences I enjoyed 
  while a student, including the opportunity to teach beginning computer 
  science students the basics of UNIX and OpenWindows (something I wish I 
  had been taught when I first started), and the ability to present some 
  of my work at the University of Puerto Rico, Mayag\"{u}ez Campus.
\end{quote}

\noindent
Dr. Michael Gelfond
\begin{quote}
  His influence, though unbeknownst to him, was one of the main reasons for 
  my return to UTEP and computer science after my extended leave from school
  while island hopping in the navy.  He taught me many things about computer
  science---and life.  Among the many things he showed me was that there 
  really is science in computer science.
\end{quote}

And finally, I must thank my dear wife for putting up with me during the 
development of this work with continuing, loving support and no complaint. 
I do not have the words to express all my feelings here, only that I love you, 
Yulia!

\vfill
\noindent
NOTE: This thesis was submitted to my Supervising Committee on the May 31, 1996.

% abstract.tex (Abstract)

\addcontentsline{toc}{chapter}{Abstract}

\chapter*{Abstract}


\tableofcontents
\listoftables
\listoffigures
\pagebreak

%----------------------------------------------------%
% with this we ensure that the chapter and section
% headings are in lowercase.


\renewcommand{\chaptermark}[1]{%
\markboth{#1}{}}
\renewcommand{\sectionmark}[1]{%
\markright{\thesection\ #1}}
\fancyhf{} % delete current header and footer
\fancyhead[LE,RO]{\bfseries\thepage}
\fancyhead[LO]{\bfseries\rightmark}
\fancyhead[RE]{\bfseries\leftmark}
\renewcommand{\headrulewidth}{0.5pt}
\renewcommand{\footrulewidth}{0pt}
\addtolength{\headheight}{0.5pt} % space for the rule
\fancypagestyle{plain}{%
\fancyhead{} % get rid of headers on plain pages
\renewcommand{\headrulewidth}{0pt} % and the line
}
\pagebreak
\pagenumbering{arabic}
\setcounter{page}{1}
\chapter{Introduction}
\label{chapter:intro}

\paragraph{Summary}


\section{Cryptography}

Cryptography is the mathematical science of protecting secret information by transforming it into a form which is illegible. Generally, the original secret information is called the plaintext while the transformed information is called ciphertext. Once the plaintext is transformed into ciphertext, it can be sent over an insecure channel to its destination. At the destination, the reverse transformation takes place yielding back the original information. Transformation of plaintext to ciphertext is referred to as encryption, while the reverse process is called decryption.

This is just one application of cryptography in modern security systems, namely providing confidentiality to secret information; though it was the application with which the idea of cryptography was first conceived. Modern cryptography is capable of also providing message integrity (with or without providing confidentiality), authentication between two parties (by providing guarantee that you are talking to the right party), non-repudiation (implying a party cannot deny the fact that it has carried through a transaction, if it has really done so) etc. 

In the late 19th century, Kerckhoff published a very important principle which has become the basis of modern security systems using cryptography \cite{kerckhoff}. He said that the security of a cryptographic algorithm must rest solely in the secrecy of its key, not in the secrecy of the algorithm itself. In other words, while designing a security system with cryptography, all cryptographic algorithms must be openly published, while the only secret parameter should be the keys used by the algorithms. This helps in the cryptanalysis of the algorithms and weaknesses can be found out by researchers and cryptanalysts. Though this is a very important principle, still design of some security systems continue to be based on propreitary algorithms. More information on certain secret algorithms and successful reverse-engineering techniques is provided in section \ref{sec:hitag2-background}.

% symmetric and asymmetric key cryptosystems

Two kinds of cryptosystems exist, depending on the manner in which keys are used: symmetric-key and asymmetric-key. In symmetric-key cryptography, a single key is shared between two communicating parties, such that if one party encrypts a message using the shared key, the other party is able to decrypt the ciphertext using the same key. In such systems, the secrecy of the shared key becomes extremely important. Also, if the number of parties grows, the number of keys required in the system becomes large, since each communicating pair would hold a different key. In addition, sharing keys between parties requires secure protocols. 

Symmetric-key cryptography was the only known kind of encryption before 1976, when Diffie and Hellman introduced the concept of asymmetric-key cryptography \cite{diffie1976ndc} for the first time. In asymmetric-key cryptography, the encryption and corresponding decryption are performed with different keys. Every party holds two keys: one public and one private. If \emph{Alice} wants to send a message to \emph{Bob}, she will encrypt the message with the public key of \emph{Bob}. At the other end, \emph{Bob} would decrypt this ciphertext using his private key. The public-private key pair are related, but it is unfeasible to determine \emph{Bob}'s private key from his public key. Since every party holds two keys, the total number of keys is reduced considerably when compared to symmetric-key cryptosystems. Also, there is no overhead in sharing the keys, as the public keys are distributed through public channels.

% explain block cipher
Symmetric-key algorithms are further classified into two different types, based on the manner in which plaintext is used by the encryption algorithm: block ciphers and stream ciphers. Block ciphers divide the plaintext into blocks of data, and each such block is then encrypted separately. The sizes of the input and output blocks are same, irrespective of the size of the key. Successive output blocks are then connected to each other in some way, depending on the mode of operation of the block cipher. \emph{DES} has been one of the most widely used block ciphers with block size of 64 bits and key size of 56 bits. \emph{DES} has been recently replaced by \emph{AES} as the standard, through a public competition which called for new block cipher designs to replace \emph{DES}.

% explain stream cipher
Stream ciphers on the other hand, have a different functional structure and working. Stream ciphers consist of an internal state, which is used in deriving one output bit through a output function. The internal state changes over successive clock cycles using a defined update function, consequently producing a stream of bits at the output. The plaintext bits are then bitwise \emph{xor}'ed with the output bits, to produce the ciphertext. The working of stream ciphers is explained in much detail in section \ref{sec:stream-cipher}. 

\section{Stream cipher}
\label{sec:stream-cipher}

\subsection{One-time pads} 
\label{sec:one-time-pads}

Before we explain the working of stream ciphers, it is important that we understand the motivation for the design of stream ciphers. An inspiration behind practical stream ciphers of today has been the one-time pad, which is also known as the Vernam cipher. In the one-time pad, the length of the key is required to be equal to or greater than the length of the plaintext. Each plaintext bit is operated with the corresponding key bit using the \emph{exclusive-or} (or \emph{xor}) operation, thus resulting in one bit of the ciphertext. So, if $p_i$ represents the $i^{th}$ bit of the plaintext, and $k_i$ represents the $i^{th}$ bit of the key, then the $i^{th}$ bit of the ciphertext is given by $c_i$ = $p_i \oplus k_i$. The corresponding decryption takes place as $p_i$ = $c_i \oplus k_i$. This is shown in the figure \ref{fig:one-time-pad}.

\begin{figure}[ht!]
	\centering
		\includegraphics[width=4.4in]{./figures/one-time-pad.PNG}
	\caption{One-time pads}	
	\label{fig:one-time-pad}
\end{figure}

The key bits are required to be completely random, without any statistical correlation between them. With this pre-condition, there is no way that an adversary could determine the secret key just by knowing the ciphertext. Since the key bits are derived from a truly random source, there could be several combinations of the plaintext and the key which result in the given ciphertext. The attacker would, in such a scenario, never be able to determine which of the specific combination is the right one, even with infinite computing power at hand \cite{one-time-pads-link}. If the attacker knows certain bits of the plaintext, then the corresponding bits of the key can be determined. If the key is not truly random, the attacker can predict some of the remaining bits of the key and thus decipher the remaining plaintext. Hence, the key should be truly random. 
Shannon in 1949, used his notion of information theory to formally prove that one-time pads are unbreakable \cite{shannon1949cts}, and termed them as having perfect secrecy.

But for practical purposes, one-time pads have several weaknesses. These are explained in the following two points.
\begin{itemize}
\item The length of the key has to be equal to the length of the plaintext so that all the plaintext bits are encrypted. Thus, in order to encrypt a long plaintext, large number of `random' key bits would have to be generated. Managing such a huge set of key bits is not practical, since the storage and transfer need to be carried out securely. 

\item As the name of one-time pad suggests, a key can be used for just one transaction. If the same key is used for encrypting two different plaintext messages, the chances that the key being broken are extremely high. The chances are precisely zero when the key is used just for one encryption and is derived from a true random source. But, the same chances are extremely high if the key is re-used in another encryption. 
\end{itemize}

Considering the above points, rather than transmitting the key then, it is a better idea to transmit the plaintext itself through the secure channel. Clearly, one-time pads are not good enough for being deployed in practical systems, but they reveal a strong design basis for stream ciphers. The component missing between one-time pads and stream ciphers is the pseudo-random sequence generator, which is discussed next. 

% still, one-time pad is useful. explain. 

\subsection{Pseudo-random sequence generators}
\label{sec:psrg}

Pseudo-random sequence generator (PRSG) is used to derive a seemingly random sequence of bits called the \emph{keystream}, using a small initial seed value. PRSG's are finite state machines having an internal state, which is initialized using the secret key, along with some more initialization parameters if required. There are two important functions part of the PRSG.
\begin{itemize}
\item A linear \emph{update function} is used to derive the next state using the current state.
\item A linear or non-linear \emph{output function} is used to generate an output bit from the current state. 
\end{itemize}

\begin{figure}[ht!]
	\centering
		\includegraphics[width=3.5in]{./figures/prsg.PNG}
	\caption{Internal model of Pseudo-random sequence generator}	
	\label{fig:prsg}
\end{figure}

Figure \ref{fig:prsg} shows the internal model of a PRSG. A stream of the output bits (occuring as the PRSG changes states) constitutes the keystream. It is important to note that the keystream is not truly random, but as the name of the generator suggests, it is pseudo or seemingly random. 

\paragraph{\textit{Use in stream cipher construction.}}
\label{para:stream-construction} 
We need to use secret keys of fixed length independent of the length of plaintext, in the construction of stream ciphers. Typically the key lengths are 128, 256 or 512 bits, which are considered secure by the standards of the computational power existing today. % check this and need a reference here %
The small secret key is used in the initialization of the PRSG thus generation a long keystream which replaces the long secret key in the one-time pads. The trade-off here is that the keystream is not truly random, since it is derived using the pseudo-random bit generator. As a result, the perfect secrecy of one-time pads does not apply to stream ciphers, but at the same time, the strength of the pseudo-random sequence generator becomes extremely important in determining how strong the stream cipher is cryptographically.

An overview of a stream cipher design is shown in figure \ref{fig:stream-cipher}. The secret key and other initialization parameters like initialization vector (IV) etc. are used to initialize the internal state of the pseudo-random sequence generator. Once the initial state is prepared, the pseudo-random sequence generator is ready to generate a stream of random looking bits or the keystream. Bits of the plaintext are \emph{xor}'ed along with corresponding bits from the keystream to generate a stream of encrypted bits. In effect, the plaintext and the keystream are used to produce the ciphertext.

\begin{figure}[ht!]
	\centering
		\includegraphics[width=4.5in]{./figures/stream-cipher.PNG}
	\caption{A stream cipher}	
	\label{fig:stream-cipher}
\end{figure}

At the receiver end, the pseudo-random sequence generator is initialized using the same parameters. Any change in the initialization of the pseudo-random sequence generator would result in a different keystream being produced, which would lead to incorrect decryption of the received ciphertext. Using the correct keystream, the ciphertext is decrypted to give the original plaintext. 

% security of PRSG %
As briefly mentioned before, the security of the stream cipher much depends on the security properties of PRSG. The output sequence is required to behave as a true random sequence. Two most important considerations for ascertaining adequate security are mentioned below \cite{robshaw1995sct}.
\begin{enumerate}
\item The period of the keystream should be large. This becomes extremely important when the length of the plaintext is large. If the keystream repeats, the same sequence would then be encrypting different parts of the plaintext. If the attacker has knowledge of some initial part of the plaintext, then the initial keystream encrypting the known plaintext can be recovered. The repeated keystream can then be used to decrypt part of the ciphertext which the attacker does not know. 

Even if the attacker has no knowledge of plaintext, certain properties about the plaintext can be derived from the two ciphertexts which are encrypted using the same keystream. On the other hand, the requirement on how large the period should be depends on the plaintext a particular application expects to encrypt.

\item The 
\end{enumerate}

\subsection{Linear feedback shift registers} 
\label{sec:lfsr}
The most widespread implementation of PRSG's is done using a linear feedback shift register (LFSR). Other methods for generating pseudo-random sequences exist as well. Since we are mainly going to deal with the use of LFSR based generators in this thesis, we would concentrate only on them. Other methods of generating pseudo-random sequences are linear congruence generators and non-linear feedback shift registers. For an introduction to these methods, we refer to \cite{zeng1991pbg}. 
% reason why LFSR are widely used, and why are we more interested in them %
% more info on other generators %

It is important to note that LFSR constitutes only the \emph{update function} of the PRSG. The \emph{output function} of the PRSG needs to be implemented on top of the LFSR. An LFSR consists of two important components which are a shift register and a linear feedback function.\\

\textbf{\emph{Shift Register.}} A shift register holds a fixed number of bits, and shifts each of them into corresponding adjacent positions (all in a particular direction) on every clock cycle. If the direction of the shift is taken to be rightwards, then in every clock cycle there is a new bit on the leftmost position of the register, and the rightmost bit is excluded from the register, as shown in the figure \ref{fig:shift-register}. If the $n$ bits of the shift register are represented by $s_n$, $s_{n-1}$, $\ldots$, $s_{1}$, $s_{0}$, then on every clock cycle we have the following transformations: $s_{out}$ = $s_{0}$, $s_{0}$ = $s_{1}$, $\ldots$, $s_{n-2}$ = $s_{n-1}$, $s_{n-1}$ = $s_{new}$; where $s_{out}$ is the bit which is excluded from the register and $s_{new}$ is a new bit. The application implementing the shift register decides how the $s_{new}$ bit is initialized and the manner in which $s_{out}$ is handled.

\begin{figure}[ht!]
	\centering
		\includegraphics[width=5in]{./figures/shift-register.PNG}
	\caption{A simple shift register and its state after one round of shift}	
	\label{fig:shift-register}
\end{figure}

For example, one of the uses of shift registers has been in the conversion of sequential (or serial) data to parallel data, and vice versa \cite{lfsr-link}. A sequence of bits can be stored in the shift register over a period of $n$ clock cycles, and retrieved in a parallel form in the $(n+1)'th$ clock cycle. Hence, $s_{new}$ bits are initialized from the input sequential stream and $s_{out}$ bits are not just excluded. 

For parallel-to-sequential data conversion, a parallel stream of $n$ bits is stored in the shift register in one cycle, and retrieved sequentially over the next $n$ clock cycles. The bit $s_{new}$ is not important in this case (since it is not being used later) and $s_{out}$ would hold bits of the sequential output stream.\\

%-----
% figure serial-to-parallel and parallel-to-serial conversion
%-----

\textbf{\emph{Linear feedback.}} A feedback function defines the $s_{new}$ bit as a function of one or more bits from among the $n$ bits of the shift register. If the function is linear, it said to be a \emph{linear feedback function}. Linearity is incorporated in the design generally by the use of $xor$ as the feedback function. Other boolean functions which are linear are negation, logical biconditional, tautology, and contradiction. According to the Wikipedia page on linearity \cite{linear-wiki}, 
\begin{quote}
A Boolean function is linear if A) In every row of the truth table in which the value of the function is 'T', there are an even number of 'T's assigned to the arguments of the function; and in every row in which the truth value of the function is 'F', there are an odd number of 'T's assigned to arguments; or B) In every row in which the truth value of the function is 'T', there are an odd number of 'T's assigned to the arguments and in every row in which the function is 'F' there is an even number of 'T's assigned to arguments.
\end{quote}

The above properties can be easily checked to hold in the truth table of the \emph{xor} function.

%One way of designing feedback, as mentioned above, is to define $s_{new}$ as a function of the $n$ bits of the shift register. An LFSR based on such a function is called \emph{Fibonacci LFSR}. A different type of LFSR, called \emph{Galois LFSR}, implements the linear function between some specific bits of the shift register. Thus, the shift register no longer shifts bits directly to adjacent positions, but does apply a function on some of the bits before shifting them. Such an LFSR can be seen in figure ??. It is important to note here that we are going to concentrate only on Fibonacci LFSR's in this thesis. As a result, in the remaining part of this thesis, the use of LFSR would refer to a Fibonacci LFSR, unless otherwise quoted.
%
%-----
% figure Galois LFSR
%-----

A simple example of an LFSR is shown in figure \ref{fig:lfsr-example1}. Let us examine the linear feedback function more closely for this example. As can be seen, certain selected bits from the shift register are \emph{xor}'ed, and the resulting value is assigned to the leftmost bit only in the next clock cycle (in addition to the shifting of bits). As a result, the internal state of the LFSR is changed. If we number the bits as \emph{1, 2,} $\ldots$ $, n$, then the bits 11, 13, 14 and 16 are used in the feedback function. These bits are referred to as \emph{tap sequence} or $taps$ of the LFSR. In general, the outputs that effect the input of the LFSR are called $taps$.\\


\begin{figure}[ht!]
	\centering
		\includegraphics[width=5in]{./figures/lfsr-example.PNG}
	\caption{A simple LFSR of size 16 bits and tap bits 11, 13, 14 and 16}	
	\label{fig:lfsr-example1}
\end{figure}

\noindent \textit{\textbf{Maximal length tap sequence.}} Based on this tap sequence, the complete cycle of various states of the LFSR is determined. Depending on the initial state, the LFSR would run through a set of states, and then return back to the initial state. This forms a cycle of states. In order to use the LFSR for generating a pseudo-random sequence, we want to have the longest cycle of the LFSR states. In other words, we want the LFSR to traverse through all possible states, which is equal to $2^n$, where $n$ is the number of bits in the shift register. This is important since it is desired that the pseudo-random sequence has a large period, as mentioned in the requirements for PSRG in section \ref{para:stream-construction}.

It is important to note that if the LFSR is in a state where all the bits are \textbf{0}, then the next state would also be the same. This condition occurs because the \textit{xor} of any number of \textbf{0}'s is always \textbf{0}. Such a state is called the \textit{trivial state}. As a result, no cycle is formed if the initial state is the trivial state. Hence the maximum number of different states in the LFSR becomes $2^n-1$. A tap sequence which generates the particular cycle of states such that the period of the LFSR is $2^n-1$, is called the maximal length tap sequence. Hence it is desirable that the LFSR contains a maximal length tap sequence, so that the largest period is obtained. The tap sequence for the LFSR in figure \ref{fig:lfsr-example1} is a maximal length tap sequence. In order to verify maximal length tap sequences, the polynomial representation of the tap sequence is considered.\\

\noindent \textit{\textbf{Polynomial representation of tap sequence.}} Once we have a tap sequence, we can express it in the form of a polynomial with modular 2 coefficients. For the example shown in figure \ref{fig:lfsr-example1}, the polynomial representation of the tap sequence is

\begin{center}
$p(x)$ =  1 + $x^{11}$ + $x^{13}$ + $x^{14}$ + $x^{16}$
\end{center}

The powers of $x$ represent the tap bits. The \textbf{1} in the starting is a result of $x^0$ and represents the $s_{new}$ bit, since $s_{new}$ can be interpreted to be existing at the \textbf{0}'th position of the shift register. $s_{new}$ becomes $s_1$ in the next clock cycle, but in the current clock cycle, $s_{new}$ is $s_0$ (which exists just theoretically). It is important to note that the polynomial representation should not depend on the numbering of the bit positions. If the same LFSR is numbered in the fashion as shown in figure \ref{fig:lfsr-example2}, then we have the following polynomial representation.

\begin{align*}
p'(x) &= x^{17} + x^{6} + x^{4} + x^{3} + x\\
p'(x) &= x*(x^{16} + x^{5} + x^{3} + x^{2} + 1)\\
p'(x) &= x^{16} + x^{5} + x^{3} + x^{2} + 1
\end{align*}

\begin{figure}[ht!]
	\centering
		\includegraphics[width=5in]{./figures/lfsr-example-reverse.PNG}
	\caption{A different numbering of the bit positions for the same LFSR}	
	\label{fig:lfsr-example2}
\end{figure}

Since the important polynomial in the second equation is the one inside the brackets, we rewrite $p'(x)$ after ignoring the factor $x$ in the last equation. The polynomial $p'(x)$ must be the same as the polynomial $p(x)$. This is also mentioned on the Wikipedia page for LFSR at \cite{lfsr-wiki} stating,

\begin{quote}
Once one maximal tap sequence has been found, another automatically follows. If the tap sequence, in an $n$ bit LFSR, is [n,A,B,C,0], where the \textbf{0} corresponds to the $x^0$ = 1 term, then the corresponding 'mirror' sequence is [n,n-C,n-B,n-A,0].
\end{quote}

In our example, the tap sequence [16,14,13,11,0] has a mirror tap sequence [16,5,3,2,0], which are represented by the polynomials $p(x)$ and $p'(x)$. A given tap sequence is a maximal length tap sequence if its polynomial representation is a \emph{primitive polynomial}. A primitive polynomial is irreducible by any other polynomial, except \textbf{1} and itself. 

\subsection{Some stream ciphers}

\section{The HiTag2 stream cipher}
\label{sec:hitag2}

\subsection{Background}
\label{sec:hitag2-background}
HiTag2 is a stream cipher originally designed by Philips Semiconductors (now NXP Semiconductors) and used for mutual authentication between a car remote control and controller in the car. Such systems (called remote keyless entry or RKE systems) have been widely implemented in modern electronic cars, and facilitate unlocking of car doors upon just a button press from within a certain maximum distance from the car. The HiTag2 stream cipher provides two-way authentication, authenticating each of the two entities to each other one by one. But, this is just one use of the cipher, and other applications could use the cipher for authentication as well as encryption. It is important to understand here that if a stream cipher is used for authentication, the keystream bits are sent directly, since the challenge/response nonces are sent in plain previously. Hence, \emph{xor} of the nonce and keystream would hide nothing. While when encryption is desired, the message bits are \emph{xor}'d with the keystream bits before sending over the public channel. 

HiTag2 was kept secret by Philips until it was reverse-engineered using the cipher's software implementation. The microcode (or assembly code) of HiTag2 was decompiled from the implementation available in the car remote controls and the RKE receivers. The C code for the cipher is availabe on the website \cite{hitag2-code}. A full system level specification of the RKE system by Philips (titled Active Tag and IC) implementing the HiTag2 cipher is available \cite{active-tag-datasheet}. HiTag2 now is the intellectual property of NXP, but the specification document holds the name of Philips since NXP did not come into existence at that time in 2005. Also, since the cryptography (HiTag2) on the RKE system was kept secret at that time, HiTag2 modes are just mentioned without any further information on the functional working of the cipher.

The design of HiTag2 is very similar to a recently reverse-engineered \cite{NohlESP-2008-usenix} stream cipher called Crypto-1, which is used in Mifare Classic smartcards by NXP. The Mifare family comprises of contactless smart cards and card readers, with a variety of different features including memory availability on the cards to store data and read/write permissions on the cards. As a result of these features, Mifare system is widely used in applications requiring more than just authentication. Some such applications are ticketing system for public transportation (OV-Chipkaart in Netherlands, Oyster cards in UK etc.) and access control in buildings. Crypto-1 cipher used in Mifare Classic cards and readers was proprietary until recently when Nohl et. al. \cite{NohlESP-2008-usenix} reverse-engineered the algorithm from its silicon implementation.

In addition, the same Mifare Classic cards and readers are shown to be easily clonable \cite{dekoninggans2008pam} by researchers at Radboud University. They have analyzed the communication protocol between Mifare Classic cards and readers, and have been been able to show that the there are several weaknesses in the Crypto-1 cipher. They have been able to recover the necessary keystream by analyzing the protocol messages. The details of this protocol analysis complements the work by Nohl et. al., and using both the ideas, mounting brute-force attacks on the cipher have been made possible. 

Apart from reverse-engineering from software or hardware implementations, a black-box reverse-engineering has also been applied to break secret algorithms. The DST RFID tag (using the DST stream cipher) was broken in 2005 using black-box reverse-engineering \cite{bono2005sac}, i.e. by selecting certain input formats and using both the input and output combinations to deduce the structure and working of the cipher. The most common application of Texas Instrument's DST tag is in vehical immobilizer systems and electronic payment systems \cite{dst-rfid-analysis}. 

Next, we describe the working of the HiTag2 cipher, and after that we carry out a brief security analysis of the cipher.

\subsection{Cipher Description}
The various components of the HiTag2 stream cipher are outlined below. 
\begin{itemize}
\item 48 bit key
\item 32 bit serial ID
\item 32 bit initialization vector (IV)
\item 48 bit internal state with linear update function (basically, an LFSR)
\item Non-linear output function based on multiplexor, with fixed data bits and address bits depending on the current internal state
\end{itemize}

The entire setup of the keystream generator is done in two phases. The first phase is the initialization of the LFSR, which sets the internal state to a non-zero value using the key and the serial ID. The second phase is the setup of the LFSR during which the LFSR is clocked (resulting in the shifting of bits) and the key and IV bits are used in the process. Once the LFSR is set, the keystream is ready to be generated from the internal state right away. We describe these phases in detail here.\\ 

\textit{\textbf{1. LFSR Initialization.}} The initialization step of the LFSR is straightforward. Instead of initializing the LFSR with zeros, the bits are initialized with the serial ID and initial part of the key. The 32 bits of the serial ID (from \emph{lsb} to \emph{msb}) are stored in the first 32 bits of the LFSR (bits from index 0 through index 31). Then, the first 16 bits of the key (again, from \emph{lsb} to \emph{msb}) are stored in the remaining 16 bits of the LFSR (bits from index 32 through index 47). With this the initialization of the LFSR is complete. This is also shown in figure \ref{fig:hitag2-1}.\\

% NEEDED - change this figure.

\begin{figure}[ht!]
	\centering
		\includegraphics[width=5in]{./figures/hitag2-1.PNG}
	\caption{LFSR Initialization}	
	\label{fig:hitag2-1}
\end{figure}

\textit{\textbf{2. LFSR Setup.}} During the setup of the LFSR, the 32 bits of IV are used with the remaining 32 bits of the key. The bits in the LFSR are shifted to the right in every clock cycle, and a new value is stored in the leftmost bit (refer section \ref{sec:lfsr}). In every clock cycle, an xor of three bits: one bit from the remaining part of the key (bits from index 16 through 47), one bit from the IV (from \emph{lsb} to \emph{msb}) and the output bit from the non-linear output function, is stored at the leftmost bit of the LFSR. After 32 clock cycles, the internal state is now prepared for keystream generation. This is shown in figure \ref{fig:hitag2-2}.\\

% NEEDED - change this figure.

\begin{figure}[ht!]
	\centering
		\includegraphics[width=5in]{./figures/hitag2-2.PNG}
	\caption{LFSR Setup Phase}	
	\label{fig:hitag2-2}
\end{figure}

\textit{\textbf{Output function.}} The output function consists of two levels of sequentially arranged multiplexors. The multiplexors have pre-decided data bits, where as the address bits are chosen either from the LFSR or from other multiplexors. In the first level, 20 bits from the LFSR are used as address bits to five four-bit multiplexors, giving a total of five bits of output (one from each multiplexor). These five bits are then used as address bits to the single five-bit multiplexor at the second level. The output bit of this multiplexor gives the stream cipher's keystream bit. In all, six instances of multiplexors cover the two levels of the output function.

Three different multiplexor functions are used to realize these six instances, and these are called $f_a^4$, $f_b^4$ and $f_c^5$. While $f_a^4$ and $f_b^4$ take in four bits for addressing the output, $f_c^5$ takes in five bits. The multiplexors $f_a^4$ and $f_b^4$ are used in the first level, where as $f_c^5$ is used in the second level. The six different instances of these functions used are described in the table \ref{tab:muxs}. In the table, the input bits for the first level of multiplexors are indicated by indices of LFSR bits.

\begin{table}[ht!]
\begin{center}
\small{
\begin{tabular}{|p{2.2cm}|l|p{2cm}|p{2.8cm}|p{1.5cm}|}
\hline 
\textbf{Multiplexor instance}	& \textbf{Function}		& \textbf{Data bits}	& \textbf{Address bits (input)}		& \textbf{Output bit}\\ \hline \hline
\multicolumn{5}{|c|}{Level 1 Multiplexors}\\ \hline \hline
MUX 1 			&	$f_a^4$			& 0x2C79			& 1, 2, 4, 5							& $o_1$\\
MUX 2 			&	$f_b^4$			& 0x6671			& 7, 11, 13, 14						& $o_2$\\
MUX 3 			&	$f_b^4$			& 0x6671			& 16, 20, 22, 25					& $o_3$\\
MUX 4 			&	$f_b^4$			& 0x6671			& 27, 28, 30, 32					& $o_4$\\
MUX 5 			&	$f_a^4$			& 0x2C79			& 33, 42, 43, 45					& $o_5$\\ \hline \hline
\multicolumn{5}{|c|}{Level 2 Multiplexor}\\ \hline \hline
MUX 6 			&	$f_c^5$			& 0x7907287B	& $o_1$, $o_2$, $o_3$, $o_4$, $o_5$		& $k_i$\\ \hline
\end{tabular}}
\end{center}
\caption{Multiplexors making the output function}
\label{tab:muxs}
\end{table}

Hence, as mentioned previously, the output bit from the function $f_c^5$ is \emph{xor}'ed with the key and IV bits for 32 clock cycles and written to the rightmost bit of the LFSR.

The source code for the initialization and setup of the LFSR is shown below in the listing. Here, the function \emph{hitag2\_output} stands for the output function, which we have just described.\\

% NEEDED - explain a little about the implementation before the code snippet. 

\lstinputlisting[frame=tb, caption={HiTag2 initialization code snippet}]{./code-snippets/hitag2_init.c}

\textit{\textbf{3. Keystream Generation.}} The output from the function $f_c^5$ constitutes the keystream. In addition, the LFSR state is changed in every clock cycle through the update function. The internal state is updated linearly, in the following fashion: an xor of the taps of the LFSR (which are bits 0, 2, 3, 6, 7, 8, 16, 22, 23, 26, 30, 41, 42, 43, 46 and 47) is written to the leftmost bit of the LFSR. Once the internal state is changed, the output bit gets recomputed, generating the next bit of the keystream. This is shown in figure \ref{fig:hitag2-3}. Following is the code snippet for the function implementing one round of the HiTag2 (\emph{hitag2\_round}), by first updating the state and then calling the output function (\emph{hitag2\_output}).\\

\lstinputlisting[frame=tb, caption={HiTag2 keystream generation code snippet}]{./code-snippets/hitag2_round.c}

% NEEDED - change this figure.

\begin{figure}[ht!]
	\centering
		\includegraphics[width=5in]{./figures/hitag2-3.PNG}
	\caption{Keystream Generation}	
	\label{fig:hitag2-3}
\end{figure}

\subsection{Brief security analysis}
\label{sec:hitag2-security-analysis}
% a basic cryptanalysis of the cipher - maximal period? linearity? %

\chapter{Time-memory tradeoff}

\paragraph{Summary}


\section{Time-memory tradeoff attack}

\textit{\textbf{Brute force attack.}} A brute force attack on a block cipher would be to try out all the possible keys which could be used to encrypt certain known (or chosen) or unknown plaintext. The key which decrypts the ciphertext to give the known plaintext or most sensible plaintext (if it is unknown) is then the original key. Though very simple in theory, brute force attacks require a very long time to break ciphers in practice. There is no storage required during this attack, but the time required to break the cipher is very long. Though modern computers have advanced tremendously in their computational speed over the last some years, design of new ciphers have incorporated longer key sizes to protect against brute force attacks, making brute forcing even more impractical.

For example, in order to break a 32 bit key, we would need to carry out $2^{32}$ decryptions on available ciphertext. Using a 2GHz processor (which is quite common for personal use), we can run $2^{31}$ clock cycles in one second (as 1 giga is $2^{30}$). Since one encryption would take a fixed number of clock cycles, say a modest $2^3$ cycles, by simple calculation, we can have a brute force attack on the cipher in $2^4$ seconds, or 16 seconds. As can be seen, this is a dismally weak key size. For a key size of 48 bits, the brute-force would take $2^{20}$ seconds which is 1048576 seconds, or just more than 12 days. In modern ciphers, the key sizes starting from 128 bits in length are considered safe. AES uses a minimum key size of 128 bits, which can be extended to use 256 bits. So, just to give a feeling of the security of a 128 bit cipher, it would take an order of $10^{30}$ years to brute force the key.\\

\textit{\textbf{Precomputed ciphertext attack.}} Brute-force attacks are just one side of the coin. The other way of breaking a cipher in a known (or chosen) plaintext attack is to precompute ciphertexts corresponding to all the possible keys and to store the (key, ciphertext) pair in a table in memory. During the attack, the attacker just needs to do a table lookup for the available ciphertext to find the corresponding key. Again, the concept is quite straightforward theoretically, but it also faces the same problem as brute-force attacks, but from a different perspective. A table lookup takes constant time (if efficient hash tables are implemented), so practically the attack time is very less. But, the amount of precomputed data is tremendous and the attacker would need very large amount of memory to save this data for the attack phase. 

Let us take the weaker case of a 32 bit key, which is far from use in today's ciphers. For each of $2^{32}$ possible keys, we need to store 32 bits of key and 32 bits of ciphertext (assuming the plaintext is 32 bits, which could very well be more). This amounts to 64 bits or 8 bytes of data for every possible key. $2^{32}$ such entries would require $2^{32}$ * 8 bytes which is 32 gigabytes. For a random access memory, 32 GB is a high requirement. For higher key sizes, this requirement gets more towards impossible.\\

\textit{\textbf{In between time and memory.}} The technique of time-memory tradeoff is a way between the above two extreme and practically difficult ideas. TMTO solves our problem by using memory in order to reduce the time taken for attack, bringing the requirements for time and memory within  practical domains. As a result, with considerable precomputation, the attack time can be reduced and so are the computational resources required.

Before we go into more details of the working of time-memory tradeoff attacks on ciphers, we would take a simple example of a general application of such tradeoffs. The example has been taken from \cite{stamp2003out}. \\

\textit{\textbf{Simple example.}} Consider the problem of finding the number of \textbf{1}'s in the binary expansion of a non-negative integer $x$ (which takes $4$ bytes). The simplest algorithm to solve this problem would for 32 operations, pick the value of the least significant bit, add it to a global \textit{sum} and shift the integer rightwards by one bit. The \textit{sum} would then be the desired result. The pseudo-code for such an algorithm is shown below.

\begin{verbatim}
// ones_count(x)
sum = 0
for i = 0 to len(x) - 1
    sum = sum + (x >> i) & 1
next i
return sum
\end{verbatim}

Here, $>>$ denotes the right shift operation and \& denotes bitwise binary \emph{and} operation. The algorithm performs 32 operations (based on the length of the integer) and has nearly no memory requirement. The other approach to solve this problem would be to store the required \emph{sum} for each of the possible $2^{32}$ integers in memory. This way, just one memory lookup is required to find the result for $x$. While in this case, we need to have a memory of the order of $2^{32}$. 

One middle way between both these approaches would be to store the \textit{sum} for all possible 8 bit numbers, rather than doing it for all 32 bit numbers. Then the memory required would be of the order of $2^{8}$. Then, we break the 32 bit integer into four bytes, and add the stored \textit{sum} for each of the four bytes, by looking up in the table. If $y_1$, $y_2$, $y_3$ and $y_4$ are the four bytes, such that

\begin{flushleft}
$y_4$ = ($x$ $\&$ $0$xFF)\\
$y_3$ = ($(x >> 8)$ $\&$ $0$xFF)\\ 
$y_2$ = ($(x >> 16)$ $\&$ $0$xFF)\\ 
$y_1$ = ($(x >> 24)$ $\&$ $0$xFF)\\
\end{flushleft}

and if p is the array which stores the \textit{sum} for all possible bytes, then the desired \textit{sum} for $x$ can simply be calculated by

\begin{center}
\large{$sum = p[y_1] + p[y_2] + p[y_3] + p[y_4]$}\\
\end{center}

The number of operations in this case is 4, as there are four lookups made to the precomputed array $p$. This is just one way of realizing a middle way between the parameters of time and memory. If the algorithm stores 4 bit values, with their corresponding \textit{sum}, then the number of operations would be 8. Hence a optimal combination of memory and time can be chosen based on the resources at hand and the application. 

% IMP: do you need to show the table here for different combinations of time and memory? %

\section{Background}
\subsection{Birthday paradox}
\label{sec:bday-paradox}

\textit{\textbf{General birthday paradox.}} Birthday paradox (or birthday problem or surprise, as it is also called), refers to the fact that in a room of 23 people, two people have the same date of birth with a probability greater than one-half. While there are 365 different possible birthdays in a year (excluding the leap year), the birthday paradox looks surprising and non-intuituve at the first glance. But, the figure has been derived from probability theory and is proved. 
% IMP: need a reference here %

A generalized definition of the birthday paradox for can be formulated as follows: given $n$ random integers where each could have $m$ different possible values, the probability that two of them would have the same value is given by the following equation \cite{menezes}.

\begin{center}
\large{$P(m,n)$ $\approx$ $1 - e^{-{n^2}/{2m}}$}
\end{center}

If we replace $m$ by 365, and $n$ by 23 in the above equation, then it can be checked that $P(365,23)$ $\approx$ $0.507$. In addition, $P(365,n)$ rapidly increases as $n$ increases, and the chances of two people having the same birthday becomes nearly $99 \%$ for $n$ = 57. If $m$ is considered to be very large, such that $m \rightarrow \infty$, then the above equation can be reduced, giving us the following condition for chances of nearly $100 \%$. 

\begin{center}
\large{$n$ = $\sqrt{\frac{\pi}{2} \times m}$}
\end{center}

The general birthday paradox is used in cryptography in determining collisions in hash functions. Consider a hash function $H$ with $h$ bits of output. The total number of different outputs the function could produce is $2^{h}$. A collision is said to occur when the hash function produces the same output for two different inputs, i.e. $H(x_1)$ = $H(x_2)$ when $x_1$ $\neq$ $x_2$. Then, according to the birthday paradox, the chances that a collision would occur are close to $100 \%$ after the hash function has produced outputs for $2^{h/2}$ random inputs (if we have $m = 2^h$ and we ignore the factor of $\frac{\pi}{2}$).\\

\textit{\textbf{Variant of the birthday paradox.}} A variant of the birthday paradox (\cite{GeneralizedAttack}) is especially more interesting to us, since it is directly used in TMTO attacks. If we consider two groups of people now instead of just one, then just 17 people are required to be present in each group, so that two people, one from each group, share the same birthday.  

We can generalize the above situation in the following way. We have two groups of random elements each having different number elements, say $n_1$ and $n_2$. The elements are non-negative integers, with both the groups having the same range $m$ for all the integers. According to \cite{menezes}, the probability of at least one coincidence in such a case is given by,
\begin{center}
\large{$P(m, n_1, n_2) = 1 -(1 -\frac{n_2}{m})^{n_1}$}
\end{center}

For the condition that $m \rightarrow \infty$, the above relation is reduced to,
\begin{center}
\large{$P(m, n_1, n_2)$ $\approx$ $1 - e^{-{n_{1} n_{2}}/{m}}$}
\end{center}

For there to be at least one coincidence, the probability must be $1$. Replacing $P(m, n_1, n_2)$ by $1$ and taking the limit $m \rightarrow \infty$, we get the following condition.
\begin{center}
\large{$n_1 \times n_2$ = $\frac{\pi}{4} \times m$ (or)\\}
\large{$n_1 \times n_2 \geq m$}
\end{center}

The last equation above is the birthday paradox we would be using the analysis of most of the TMTO attacks on HiTag2.

% Add the case of handbook statement, with or without replacements?
% also briefly how this paradox would be applied to the TMTO?

% -------------
% what the simple birthday paradox is %

% generalize it in terms of n %

% introduce birthday attack, define scenario (then take hash as example) %

% what is the variant of birthday paradox %

% meet in the middle attack on DES by diffie hellman %
% a TMTO attack would always use variant of birthday attack %

% Question - How can the variant of the birthday problem be derived from the birthday problem?
% Question - Probability of one-half on 23, or is it one? root(pi * m / 2)
% Question - Is meet in the middle attack a TMTO attack? (I think NO!)
% --------------

\subsection{Hash tables}
\label{sec:hash-tables}

Hash tables are used during the precomputation phase as a data structure to store the (prefix, state) pairs in memory. The advantage that hash tables offer is in terms of the search time. The search time provided by hash tables in the best case and the average case is close to $O(1)$, while the worst case search time is $O(n)$ occuring with very less probability. A very important role is played by the hash function chosen for the table. 

The basic data structure for hash tables consists of a pair of data elements, one for holding the \emph{value} to be stored, and the other which acts as a unique identifier for that value, called the \emph{key}. The hast table is basically an array of such (key, value) pairs. Typically in a normal array, pairs would be stored starting from the initial index of the array, with the index increasing with every pair. In a hash table, the pairs are not stored consecutively, but in an order which would make searching them more efficient at a later stage. The index at which a particular pair is stored depends on the key and the hash function. A hash of the key is calculated, and if required, reduced to the domain of the indices. The pair is stored at that index. 

While retrieving a value, the corresponding key is provided and hash of the key is calculated, giving the index of the pair. The index is used to retrieve the value from the array, which is a constant time operation. So, in the ideal scenario, a hash table can provide a constant time search algorithm. But due to the problem of collision, the search time increases by a certain factor. Collision occurs, if the same hash value (thus the same index) is computed for two different keys. In such a case, two different (key, value) pairs would contend to be stored at the same index, thus colliding.

Several proposals have been suggested to avoid the problem of collision. The most popular among them are linear probing and separate chaining. We discuss separate chaining here, since we have implemented this scheme for the two attacks in this section. The basic idea behind separate chaining is that if there is a collision at a particular index, a separate chain holding all the pairs for that index be created. In addition, a reference to the chain must be stored at that index in the array. During retrieval, the index is computed and the reference to the chain is taken from that index. At the reference location, the required value is retrieved by searching through the entire chain. 

The implementation of the hash table is done by Christopher Clark and has been taken from the web source \cite{hash-table-impl}, with due acknowledgement. 


\section{Babbage-Golic attack}
Time-memory tradeoff attacks on stream ciphers are relatively new than those on block ciphers. The first TMTO attack on block ciphers was published by Hellman in 1980. In contrast, the first and the simplest TMTO attack on stream ciphers was published by Babbage \cite{babbage} and Golic \cite{golic}, independent of each other in 1995 and 1997 respectively. We would call this as the BG attack and explain it here. 

As discussed in section \ref{sec:psrg}, PRSG generates a long keystream using a short secret key, to which the bits of the plaintext are xor'ed giving the ciphertext bits. In this way, a stream cipher is produced. A simple model of the PRSG taken from \cite{babbage} is shown in figure \ref{fig:psrg-model}. The initial state of the PRSG is $S_0$, which is derived using the secret key and other initialization parameters. The consequent states are indicated by $S_i$. The right arrow represents the update function or the state transition function, while the downward arrow represents the output function of the PRSG.

\begin{figure}[h]
	\centering
	\includegraphics[width=3in]{./figures/prsgmodel.png}
	\caption{Model of pseudo-random sequence generator (PRSG)}	
	\label{fig:psrg-model}
\end{figure}

If the size of each state is $n$ bits, then the maximum number of different states would be $2^n$. As a result, we have $2^n$ different keystream bits, before the keystream starts repeating due to repeating states. The keystream bits are represented by $k_0$, $k_1$ $\ldots$, $k_{m-1}$, where $m$ = $2^{n} - 1$ (which excludes the trivial state containing all \textbf{0}'s). The goal of the attacker is to find at least one value of the internal state which occurs during the generation of the keystream. The attacker could then move the PRSG forward to generate more keystream (if the keystream is limited). The found internal state could also be used to run the PRSG backwards and find the initial state. From the initial state, finding the key is not a difficult task if the initialization algorithm and other parameters are known.\\

\textit{\textbf{Prefix of the output sequence of states.}} If the current state of the PRSG is $S_r$, then an infinite output sequence can be generated by clocking the PRSG from that state. The first $p$ bits of this output sequence is called the prefix of state $S_r$ and would be represented by the bits $k_r$, $k_{r+1}$ $\ldots$, $k_{r+p-1}$. Each of the $2^n$ possible states of the PRSG would have such output sequence and thus prefixes, which usually are unique to that particular state if $p$ is greater than or equal to $n$ \cite{biryukov2000rtc}. If the size of the prefix is less than $n$, then there would be many states which could possibly generate this prefix. For example, if the size of the state is 48 bits and we consider prefixes of size 32 bits, then the total number of states which generate this prefix is about $2^{48} \times 2^{-32}$ = $2^{16}$. If the size of the prefixes is increased to 48 bits which is also the size of the state, then there is usually just one state generating the prefix. An advantage of this property is taken in the BG attack.\\

\textit{\textbf{The attack.}} We would have two phases in the attack just as in any TMTO attack. The first phase is the precomputation phase and the second phase is the attack phase. During the precomputation phase, the attacker would randomly select $n_1$ values out of the $2^n$ values the PRSG state could have. For each of these states, the prefix is computed, and the (prefix, state) pair stored in memory. The data structure that needs to be used for storing the pairs is a hash table. More details about hash tables are provided in section ??.

During the attack phase, the attacker is assumed to have access to some part of the initial keystream. In practice, the attacker would have access to the ciphertext bits. But, we assume that the attacker knows the plaintext bits before hand by some means, and thus the keystream is calculated in the following manner, $k_i$ = $p_i \oplus c_i$, as also mentioned in chapter \ref{chapter:intro}. The attacker then selects overlapping subsequences of size $n$ from the keystream, and tries to find a match in the hash table. The first subsequence would be $k_0$, $k_1$ $\ldots$, $k_{n-1}$ corresponding to state $S_0$, the second subsequence would be $k_1$, $k_2$ $\ldots$, $k_{n}$ corresponding to state $S_1$, and so on. If such a prefix exists in the memory, then the current state is determined from the matched (prefix, state) pair.\\

\textit{\textbf{Analysis of the attack.}} Let us assume that the attacker tries $n_2$ prefixes during the attack phase, before the first hit is observed. Then, we can apply the birthday paradox here, in order to find the approximate value of $n_2$. According to the variant of the birthday paradox, explained in section \ref{sec:bday-paradox}, we have the following condition:

\begin{center}
\large{$n_1 \times n_2$ $\geq$ $2^n$}
\end{center}

Also, for sufficient number of prefixes to be available, the keystream should have a length of at least $(n_2 + n - 1)$ bits. In such a case, the attacker would have exactly $n_2$ prefixes of length $n$ each to search in the memory.


\section{Babbage-Golic attack on HiTag2}

The HiTag2 stream cipher is introduced in section \ref{sec:hitag2}. We are going to consider a variant of the BG attack to be implemented on HiTag2. The attack phase of the BG attack is going to be the same, but the precomputation phase is modified so that we have a match in the memory with certainty \cite{erik-discussions}. \\

\textit{\textbf{Precomputation phase.}} During the precomputation phase, the selection of states is not done at random now, but with certain determinism. Only an initial state is selected at random for precomputation ($S_{initial}$), while the remaining states are derived using this state. The prefix for $S_{initial}$ is first computed and the (prefix, state) pair stored in the hash table. The next state is derived using a state transition function, which returns the state $q$ states ahead of the current state. The next state then is represented by $S_{initial+q}$. Similarly, the prefix for $S_{initial+q}$ is computed and the (prefix, state) pair stored in the hash table. The process is repeated, and consequently we have a list of states (and their corresponding prefixes) in memory which are at a \emph{distance} of $q$ states from each other. These would be $S_{initial}$, $S_{initial+q}$, $S_{initial+2q}$ and so on. Until the complete circle of states is covered (after which the same states are repeated) we have unique (prefix, state) pairs stored for the precomputation phase. At the end of the precomputation phase, we would have a total of $2^n/q$ states in the hash table. The precomputation phase would take and order of $2^n/q$ of memory space.\\

\textit{\textbf{Attack phase.}} Let us assume that the initial state of the keystream available to the attacker is $S_{unknown}$ (as this state is unknown to the attacker). The attacker selects subsequences from the keystream and matches them with the prefixes stored in the memory. Then, it is our claim that if the attacker gets $q$ subsequences from the keystream (which represent $q$ unknown states), a prefix in the memory is bound to get matched with a keystream subsequence. It is because the fixed distance between any two states in the memory is that of $q$ states. Hence, in the worst case, the attack phase would take $q$ order of operations to find a match in the memory. \\

\textit{\textbf{Matrix representation of state transition function.}} As mentioned above, the state transition function is used in the precomputation phase for deriving a non-adjacent state from the current state. The design of the this function needs an important consideration, which we discuss here. The update function in LFSR can be represented by a binary matrix $U$ \cite{trappe2005icc}. The topmost row of $U$ contains \textbf{1} at positions representing the tap bits, and \textbf{0} otherwise. In the remaining rows of $U$, the values of the bits are such that the last 47 bits of the new state are the first 47 bits of the previous state. It must be noted that the indexing in the matrix is the same as that in the state register. Thus, the leftmost position in the top row of the matrix represents the \textbf{48}'th bit in the state register, and the rightmost position represents the \textbf{1}'st bit in the state register. The matrix $U$ for HiTag2 is shown in figure \ref{fig:hitag2-transition-matrix}.

\begin{figure}[h!]
	\centering
	\includegraphics[width=3.5in]{./figures/hitag2-transition-matrix.png}
	\caption{Update function matrix $U$ for HiTag2}	
	\label{fig:hitag2-transition-matrix}
\end{figure}

If the current state of the LFSR is $S_{current}$, then the next state $S_{next}$ can be computed by the matrix multiplication shown below.

\begin{center}
\large{$U . S_{current}$ = $S_{next}$}\\
\end{center}

This is represented in the matrix notation as follows.

\begin{center}
\large{
$U.$
$\begin{bmatrix}
s_{n} \\
s_{n-1} \\
\vdots \\
s_{1} \\
s_{0}
\end{bmatrix}$ = 
$\begin{bmatrix}
s_{new} \\
s_{n} \\
\vdots \\
s_{2} \\
s_{1}
\end{bmatrix}$}
\end{center}

Matrix multiplication for binary matrices is to perform a bitwise boolean \emph{and} operation (denoted by \&) and then \emph{xor} the resulting bits. In notation, if we have two rows $r_1$ and $r_2$ of size $n$ for multiplication, then we have the following,

\begin{center}
\large{$r_1$ = $r_{11} r_{12} \ldots r_{1n}$}\\
\large{$r_2$ = $r_{21} r_{22} \ldots r_{2n}$}\\
\large{$r_1 \times r_2$ = $(r_{11} \& r_{21}) \oplus (r_{12} \& r_{22}) \oplus \ldots \oplus (r_{1n} \& r_{2n})$}
\end{center}

To compute the state occuring $q$ states after the current state, we perform the above matrix multiplication $q$ times, as follows,

\begin{center}
\large{$S_{current + q}$ = $\underbrace{U . U . U \dots U}_{q} . S_{current}$}, or\\
\large{$S_{current + q}$ = $U^q . S_{current}$}\\
\end{center}

The entire precomputation phase would require $2^n/q$ states. Thus, if we have to repeat the above multiplication for $2^n/q$ states, the total number of matrix multiplications required would be $2^n$, since the computation for one state requires $q$ multiplications. In the case of HiTag2, an order of $2^{48}$ operations would be required for precomputation, which would take years to complete.

From \cite{erik-discussions}, there is a better solution for computing the states for the precomputation phase. Instead of performing matrix multiplication by multiplying with $U$ each time, a repeated square of $U$ can be performed yielding an efficient algorithm. The square of $U$ is first computed, which is squared to compute $U^4$, which is again squared to compute $U^8$, and so on. If $\log_2{q}$ is an integer, then the squaring is repeated for $\log_2{q}$ number of times, as shown below.

\begin{center}
\large{$S_{current + q}$ = $\underbrace{(((U^2)^2) \dotsc )^2}_{\log_2{q}} . S_{current}$}\\
\end{center}
\label{eq:state-trans}

As can be seen, computing each state would take $\log_2{q}$ multiplications, and thus the entire precomputation phase would take $q \times \log_2{q}$ matrix multiplications. This is much effecient than the previous complexity of $2^n$. The above equation shall then be used in the precomputation phase of this attack to compute the (prefix, state) pairs.\\

\textit{\textbf{Implementation details.}}
% implementation results %
% TMD calculations, specially data requirement for the attack %

\section{Variant of Babbage-Golic attack on HiTag2}
In the previous attack, there is a big assumption on the amount of keystream available to the attacker. The requirement for the keystream is proportional to the number of operations performed during the attack phase. Considering the use of HiTag2 in car keys, it is certain that such long keystream would not be available. Rather several short keystreams (for each transaction between the car and car key) would be available to attacker over a span of time. The difference in each such keystream would be in the initial state, arising due to a different IV every time. The secret key shared between the car and the car key would be the same, and for every transaction a new random IV is generated and used in computing the initial state. 

As explained in section \ref{sec:hitag2}, 32 bits of data is exchanged between the car and the car key, when a transaction takes place. These 32 bits of data, known as the \emph{tag}, perform the function of authenticating the car key to the car. 


\chapter{Time-memory-data tradeoff using Hellman tables}
\label{chapter:tmdto-hellman}

\paragraph{Summary}


\section{Hellman tables}



\chapter{Time-memory-data tradeoff using Rainbow tables}
\label{chapter:tmdto-rainbow}

\paragraph{Summary}


\section{Rainbow table for block ciphers}
\label{sec:rainbow-block}

Rainbow table was introduced by Philippe Oechslin in \cite{oechslin:mfc}. Rainbow table (or rainbow chains, as called by the author) is a different way of precomputing data for the attack phase of a TMTO attack. Oechslin introduced rainbow table on block ciphers with improvements over the Hellman tables. By using rainbow table, the attack time is supposed to be reduced by a factor of $2$.

In Hellman tables, merging of chains within different tables is prevented by using different reduction functions for each table. However, collisions within the same table cannot be avoided completely. Though the number of elements in each table is restricted according to the relation derived using the variant of the birthday paradox, still no guarantee can be provided that collisions would not occur in the same table. This is due to the fact that the relation is probabilistic in nature, as is the birthday paradox from which it is derived. Rainbow table reduces the problem of collisions considerably.\\

\noindent \textit{\textbf{Precomputation phase.}} The rainbow table comprises of one huge table instead of $t$ tables, with $mt$ chains each having $t$ keys. The interesting difference from Hellman tables is that instead of reduction functions being changed for tables, reduction function are changed by column. If $SP_i$ is the starting point of a chain, then subsequent keys are computed by the functions \mbox{$F_1, F_2, \cdots, F_t$}, where $F_j(K) = R_j(E_{K}(P))$ for $1 \leq j \leq t$. This is shown in the following equations. 

\begin{align*}
& & K_{i,0} & = SP_i & & & &\\
1&. &K_{i,1} & = F_1(K_{i,0}) & & & &\\
2&. &K_{i,2} & = F_2(K_{i,1}) & & & &\\
& & &\vdots & & & &\\
(t-1)&. &K_{i,t-1} & = F_{t-1}(K_{i,t-2}) & & & &\\
(t)&. &K_{i,t} & = F_t(K_{i,t-1}) & & & &\\
& & EP_i & = K_{i,t} & & & &\\
\end{align*}

For every chain, the same sequence of mapping functions from $F_1$ to $F_t$ are used in computing subsequent keys. As usual, the starting and end points for each chain are stored in a hashtable, with the end points as the hash key and the starting points as the hash value. A rainbow table is shown in figure \ref{fig:rainbow-table}.

\begin{figure}[ht!]
	\centering
		\includegraphics[width=3.5in]{./figures/rainbow-table.PNG}
	\caption{Rainbow table for block ciphers}	
	\label{fig:rainbow-table}
\end{figure}

The time for preparing the rainbow table is the same as the number of computations carried out. This is equal to $P$ = $mt \times t$ = $mt^2$. Also, the order of memory required for the hash table is $M$ = $mt$. \\

\noindent \textit{\textbf{Attack phase.}} The attack phase is a little more complicated as compared to that for Hellman tables. But, as we shall see, the number of computations required in the worst case, to find a match, is reduced by a factor of $2$. 

First, let's consider the possibility of the ciphertext appearing in the last column of the rainbow table. If this is the case, then an end point $EP_i$ would match with the reduction of the ciphertext $C$ which is $X$ = $R_t(C)$. Note that the reduction function for the $t$'th column is used here. If the match occurs, then $(t-1)$ computations are performed starting from the corresponding starting point $SP_i$, with the mapping function changing in each column. $K_{i,t-1}$ is then the required key, if it is not a false alarm. If the match does not occur, the number of operations (number of calls to any of the mapping functions $F_1$ to $F_t$) performed is $0$. 

So next, the possibility of $C$ appearing in the $(t-1)$'th column is explored. For this the value $X = F_{t}(R_{t-1}(C))$ is compared with the end points. If there is a match, then $(t-2)$ computations are performed from $SP_{i}$ using the mapping functions $F_1, F_2, \cdots, F_{t-2}$. The key $K_{i,t-2}$ computed is the required key. Otherwise, if there is no match, the number of operations performed is $1$.

This procedure is repeated for all the columns, until in a worst case scenario, $C$ happens to appear in the second column of the table. In such a case, we would iteratively call the mapping functions $F_2, F_3, \cdots, F_t$, thus amounting to $(t-1)$ operations, before $EP_j$ is matched. The total number of operations in the worst case scenario become, 

\begin{align*}
&= 0 + 1 + 2 + \cdots + (t-1)\\
&= t(t-1)/2\\
&\approx t^2/2
\end{align*}

The attack time then is of the order of $t^2/2$, thus $T$ = $t^2/2$.\\


\noindent \textit{\textbf{Tradeoff equation.}} Using the following equations, the tradeoff equation can be derived. 

\begin{align*}
M &= mt\\
T &= t^2/2\\
mt^2 &= N\\
\end{align*}
The tradeoff equation then comes to be,
\begin{align}
\label{eq:tmdto-rainbow-block} 2TM^2 &= N^2
\end{align}


\section{Rainbow tables for stream ciphers}
\label{sec:rainbow-stream}


\section{Results of implementation}
\label{sec:rainbow-impl}
\chapter{Conclusion}


\section{General comments}


\section{Future work}


\bibliography{thesis}
\bibliographystyle{plain}
\addcontentsline{toc}{chapter}{\bibname}

\appendix
\include{./appendices/AppendixA}
\include{./appendices/AppendixB}

\end{document}
