% chap3.tex (Definitions and Theorem)

\chapter{Solving Interval Linear Systems Is NP-Hard: Known Result}

In this chapter we will define formally what it means to solve a system of
interval linear equations and present a negative result related to such
systems that was first discovered by Kreinovich, Lakeyev and Noskov in the
early 1990's.

\section{Definitions}

\begin{definition}
{\rm An expression of the form
\begin{equation}
  \sum_{j=1}^n [a_j^-,a_j^+] x_j = [b^-,b^+] \label{intlineq}
\end{equation}
is called an {\em interval linear equation\/} with $n$ unknowns,
$x_1,\ldots,x_n$.}
\end{definition}

\begin{definition} \label{solutiondef}
{\rm Let
\begin{equation}
  \sum_{j=1}^n [a_{ij}^-, a_{ij}^+] x_j = [b_i^-, b_i^+] \label{d1}
\end{equation}
for $i=1,\ldots,m$ be a {\em system of interval linear equations}.  The
tuple $(x_1,\ldots,x_n)$ is called a {\em solution\/} of the system (\ref{d1})
if there exist values $a_{ij}\in[a_{ij}^-,a_{ij}^+]$ and $b_i\in[b_i^-,b_i^+]$
such that
$$
  \sum_{j=1}^n a_{ij} x_j = b_i
$$
for all $i=1,\ldots,m$.}
\end{definition}

\begin{definition}\label{solvingdef}
{\rm By {\em solving\/} a system (\ref{d1}) of interval linear equations, we
mean computing the bounds $x_j^-$ and $x_j^+$ where
$$
  x_j^-=\min\{x_j|(x_1,\ldots,x_n)
   {\rm\ is\ a\ solution\ to\ system\ (\ref{d1})}\}
$$
and
$$
  x_j^+=\max\{x_j|(x_1,\ldots,x_n)
   {\rm\ is\ a\ solution\ to\ system\ (\ref{d1})}\}
$$
for all $j=1,\ldots,n$.}
\end{definition}

To clarify the problem, consider a (hypothetical) feasible algorithm that can
be used to solve any system of interval linear equations.  Such an algorithm
would require the following input to be given:
\begin{itemize}
\item the number of equations $m$,
\item the number of unknowns $n$,
\item the endpoints $a_{ij}^-$ and $a_{ij}^+$ for all $i=1,\ldots,m$ and
      $j=1,\ldots,n$, and
\item the endpoints $b_i^-$ and $b_i^+$ for all $i=1,\ldots,m$.
\end{itemize}
Such input would define a particular system of the form (\ref{d1}).  For the
algorithm to be feasible it must be able, given any particular system
defined by the input, to compute in polynomial time the following output:
\begin{itemize}
\item the endpoints $x_j^-$ and $x_j^+$ such that
$$
  x_j^-=\min\{x_j|(x_1,\ldots,x_n)
   {\rm\ is\ a\ solution\ to\ the\ given\ system}\}
$$
and
$$
  x_j^+=\max\{x_j|(x_1,\ldots,x_n)
   {\rm\ is\ a\ solution\ to\ the\ given\ system}\}
$$
for all $j=1,\ldots,n$.
\end{itemize}

\section{Theorem}

\begin{theorem}
The problem of solving systems of interval linear equations is computationally
intractable (\/{\rm NP}-hard). \label{t1}
\end{theorem}

\medskip

\noindent
This theorem was proven (see~\cite{Kreinovich1993}) in 1993 by Kreinovich,
Lakeyev and Noskov by a reduction from SAT\footnote{Recall from chapter
\ref{NP-HARD-CHAP} that SAT is the problem of satisfiability of Boolean
formulas.}.

\section{What If Intervals Are Narrow?}
The proof of theorem \ref{t1} uses intervals that are of type $[0,1]$ that
correspond, in measurement terms, to measuring a value of magnitude $0.5$ 
with an absolute accuracy of $0.5$, or a relative accuracy of 
$100\%$. Such measurements, though possible, are not very accurate. 
Most measuring devices of practical use in science and industry are far more
accurate, and hence, intervals are much narrower.  The natural question is:


\begin{quote}
{\it If we restrict ourselves to systems with narrow intervals only, will the
problem of solving systems of interval linear equations still be\/
{\rm NP}-hard?}
\end{quote}

\noindent
This question was first analyzed by Lakeyev and Kreinovich in 1995 
(see~\cite{Lakeyev1995}).  In the next chapter we will present the result of 
that analysis.
