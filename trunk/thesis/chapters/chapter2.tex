% chap2.tex (Definitions)

\chapter{NP-Hard Problems}\label{NP-HARD-CHAP}

In this chapter we provide a brief introduction to the theory of
NP-completeness and provide the notation and definitions necessary to
accurately describe what we mean when we say that a problem is NP-hard.
The theory of NP-completeness was first developed by Cook (see
\cite{Cook1971}) and Karp (see \cite{Karp1972}), and later, independently
by Levin (see \cite{Levin1973}).  We also provide here the theorems that
form the basis for the proof of the main result of this thesis.  This material
is inspired by the well-known book {\it Computers and Intractability: A Guide
to the Theory of\/ {\rm NP}-Completeness\/} by M.~R.~Garey and D~.S.~Johnson
(see \cite{Garey1979}).  Material on NP-completeness can also be found in
basic textbook on theory of computation.

\section{What Is a Problem?}

Informally, a {\em problem\/} is a general {\em proposition\/} (i.e., question
proposed) for which we desire a satisfactory {\em solution}.
Usually, a problem possesses one or more {\em parameters\/} (i.e., free
variables) whose values are left unspecified.  We describe a problem by
defining:
\begin{itemize}
\item the problem parameters, and
\item the properties of a desired solution.
\end{itemize}
For example, finding the prime factors of a natural number $n>1$ is a
problem called {\em prime factorization}.  The problem is defined as follows:
\begin{enumerate}
\item There is one parameter, a natural number $n>1$.
\item A solution is the set $\{p_1,\ldots,p_n\}$ of all prime numbers $p_i$
      such that
      $$
        n=p_1^{m_1}\cdot\ldots\cdot p_n^{m_n}\mbox{\ \ \ \ \ \ \ \ \ \ }
      $$
      where $n>1$, $m_i>0$, and $n$ and $m_i$ are natural numbers.
\end{enumerate}

In order to specify a particular {\em instance\/} of the problem, we need to
specify the values of all problem parameters.  For example, we can specify an
instance of the prime factorization problem by providing the parameter $n=60$.  
Since $60=2^2\cdot 3\cdot 5$, the solution to this instance of the problem is 
$\{2,3,5\}$.

Informally, a {\em decision problem\/} is any general problem with only two
possible solutions: ``yes'' or ``no.''  For any decision problem $\Pi$, we
denote by $D_\Pi$ the {\em set of all instances of\/} $\Pi$, and by
$Y_\Pi\subseteq D_\Pi$ we denote the {\em set of ``yes'' instances of\/} $\Pi$
(i.e., instances of $\Pi$ where the solution is ``yes'').
For example, the problem of determining whether a number is prime is a
decision problem whereas prime factorization is not.

\begin{definition}
{\rm A {\em search problem\/} $\Pi$ consists of a set $D_\Pi$ of finite
objects called {\em instances\/} and, for each instance $I\in D_\Pi$, a set
$S_\Pi[I]$ of finite objects called {\em solutions\/} for $I$.}
\end{definition}

\begin{definition}
{\rm An algorithm is said to {\em solve\/} a search problem $\Pi$ if, given
as input any instance $I\in D_\Pi$, it returns ``no'' as a solution whenever
$S_\Pi[I]=\emptyset$, and some solution $s\in S_\Pi[I]$ otherwise.}
\end{definition}

\begin{definition}
{\rm A {\em decision problem\/} $\Pi$ is a search problem where
$$
  S_\Pi[I]=\left\{ \begin{array}{ll}
                     \{\mbox{``yes''}\} & \mbox{if $I\in Y_\Pi$} \\
                     \emptyset         & \mbox{otherwise.}
                   \end{array}
\right.
$$}
\end{definition}

\begin{definition}
{\rm For any alphabet $\Sigma$, a subset of strings $L\subseteq\Sigma^*$ is
called a {\em language over the alphabet $\Sigma$}.}
\end{definition}

\begin{definition}
{\rm An {\em encoding scheme\/} $e$ for a problem $\Pi$ is a fixed method for
mapping each instance~$I$ of $\Pi$ to some appropriate string $x\in\Sigma^*$
for some fixed alphabet~$\Sigma$.}
\end{definition}

\noindent
{\em Example}. Consider the decision problem of determining whether a given
number $n$ is prime.  Specifying the value of $n$ provides a particular
instance of this problem.  Let us now look at two different possible encoding
schemes for the problem:
\begin{enumerate}
\item An encoding scheme $e_{asc}$ that maps any instance of the problem to
      an ASCII character file (i.e., one long string of ASCII characters that
      is, e.g., the source code for a computer program representing this
      instance).
\item An encoding scheme $e_{bin}$ that maps the same instance to a binary
      file (i.e., one long string of 0's and 1's that is, e.g., the object
      code for a the computer program representing this instance).
\end{enumerate}
Notice that both encoding schemes provide a method of representing the same
instances of the same problem, even though they use different alphabets.

Since it is easy to translate any reasonable encoding into
any other reasonable encoding, we will not concern ourselves with any
particular encoding scheme in the contents of this thesis.  We require only
that an encoding scheme be reasonable (e.g., is easily decoded, does not
excessively pad, does not provide problem solution as part of the encoding,
etc.) (For a more detailed discussion of what is meant by ``reasonable'' see
\cite{Garey1979}).

\begin{definition}
{\rm For any decision problem $\Pi$ and for any encoding scheme $e$ for $\Pi$
that (for some fixed alphabet~$\Sigma$) maps to $\Sigma^*$, the {\em language
$L$ associated with $\Pi$ and $e$\/} is the set of ``yes'' instances of $\Pi$
encoded under $e$, i.e.,
$$
  L[\Pi,e]=\{x\in\Sigma^*|\mbox{$x$ is the encoding under $e$ of an instance
   $I\in Y_\Pi$}\}.
$$}
\end{definition}

\section{Turing Machines}

We now need to discuss the models of computation that we will be using,
namely Turing machines.  For a formal definition, the reader is referred to
any basic textbook on theory of computation.  Informally, a Turing machine is
a computing device that has one read/write tape (memory) and a finite
control (the program). The machine accesses the tape with a read/write head
positioned on the tape.  In one step, the program can read the symbol at the
current position, optionally write a symbol, and optionally move the head
backward or forward one position.
The result of any computation is the contents of the tape when the Turing
machine reaches a halt state.  If the Turing machine only gives ``yes'' or
``no'' answers, then the {\em set accepted by the Turing machine\/} is the
set of inputs that lead the machine to a ``yes'' answer.  The {\em time\/}
taken by a Turing machine on given input $x$ is defined to be the number of
steps the Turing machine takes to reach a halt state, starting with $x$ on
the tape.

A nondeterministic Turing machine is one that (possibly) has
more than one next configuration from the current configuration.
This leads to many possible computations from the same initial
configuration, usually modelled as a {\em tree of possible
computations}.
The language $L$ accepted by a nondeterministic Turing machine is the set of
all strings $x$ such that the Turing machine has at least one possible
computation that leads to a ``yes'' answer with $x$ as its input.
The {\em time\/} taken by a nondeterministic Turing machine with given input
$x$ is defined to be the number of steps it takes in the longest possible
computation for $x$ before reaching a halt state.
(There are actually many ways to define time for a nondeterministic Turing
machine, but for our purposes, all are equivalent.)

An oracle Turing machine is a Turing machine with the added capability that
it can query an {\em oracle\/} in order to determine set membership.  To be
more precise, let us start by choosing a set $A$ to be used as the oracle.
The oracle Turing machine has a special query tape where it can write strings
that are queries to the oracle.  After writing a string $x$ on the query tape,
the oracle Turing machine enters a special state $q_?$ called the {\em query
state\/} from which it goes automatically to a new state, either $q_{no}$ or
$q_{yes}$ depending on whether $x \in A$.  This sequence of state
transformations is defined to be only one step in a computational process.
A common way of looking at an oracle Turing machine is to see an oracle query
as if the program calls a hypothetical procedure that determines membership
in $A$, but only gets charged one step for the call.
An oracle Turing machine can be deterministic or
nondeterministic and its time is defined in like fashion.

\section{The Classes P and NP}

The complexity classes P and NP are based on polynomial-time bounded
deterministic and nondeterministic Turing machines.  By {\em polynomial
time\/} we mean that the number of steps is bounded by some polynomial of the
input size; therefore, a polynomial-time bounded Turing machine is one in
which the number of steps required before reaching a halt state is bounded by
a polynomial of the input size.

\begin{definition}
{\rm The {\em class\/} P is the set of all languages $L$ such that there
exists a polynomial-time deterministic Turing machine accepting $L$.}
\end{definition}

\begin{definition}
{\rm The {\em class\/} NP is the set of all languages $L$ such that there
exists a polynomial-time nondeterministic Turing machine accepting $L$.}
\end{definition}

The class NP was first defined by Cook in the early 1970's (see
\cite{Cook1971}).  With its introduction came one of the most important open
problems in theoretical computer science: whether P=NP.
Most computer scientists believe that P$\neq$NP.
One reason for this is that simulating a nondeterministic
Turing machine with a deterministic Turing machine appears to require
exploring the whole tree of possible computations, which would require an
exponential amount of time in (at least) some cases.  However, no proof that
P$\neq$NP is known, nor does it seem likely that one will be found in the
near future.

\section{NP-Complete, NP-Hard, and Reductions}

One important notion about the class NP is the existence of {\em complete\/}
problems. Crudely speaking, complete problems for a class are the hardest
problems in the class.  Thus, NP-complete problems are the hardest problems
in the class NP.  Also of importance is the notion of NP-hardness.
Informally, NP-hard problems are problems that are at least as hard as
NP-complete problems. Cook (see \cite{Cook1971}) proved that the problem of
satisfiability of Boolean formulas (known as SAT) is an NP-complete problem,
and Karp (see \cite{Karp1972}) showed that many other important problems
are also NP-complete by use of {\em reductions\/} (generically denoted
as $\leq_r$).  Completeness is formally defined in terms of reductions.
We will now define two types of reductions, both dealing with the notion of
polynomial time.  The {\em time\/} that an algorithm (e.g., a reduction)
requires in order to solve a problem is defined to be the number of steps as
a function of the size of the input. The size here, of course, depends on the
encoding scheme chosen. Recall that polynomial time means that the number of
steps is bounded by some polynomial of the input size. As an example, if we
say that an $n^3$ algorithm $\cal U$ exists for solving a problem $\Pi$ of
size $n$, we mean that $\cal U$ will take no more than $n^3$ steps in order
to solve $\Pi$.  Therefore, we say that $\cal U$ is a polynomial-time
algorithm.  Though not explicitly stated as being algorithms, the following
two definitions refer to polynomial-time algorithms.

\begin{definition}
{\rm A {\em polynomial-time many-one reduction\/} $\leq^p_m$ (or
{\em polynomial transformation\/}) from a language $L_1\subseteq\Sigma^*_1$
to a language $L_2\subseteq\Sigma^*_2$ (denoted $L_1\leq^p_m L_2$) is a
polynomial-time computable function $f:\Sigma^*_1\rightarrow\Sigma^*_2$ such
that
$$
  \forall x\in\Sigma^*_1\ (x\in L_1 \Longleftrightarrow f(x)\in L_2).
$$}
\end{definition}

\begin{definition}
{\rm A {\em polynomial-time Turing reduction\/} $\leq^p_T$ from a language
$L_1\subseteq\Sigma^*_1$ to a language $L_2\subseteq\Sigma^*_2$ (denoted
$L_1\leq^p_T L_2$) is a polynomial-time oracle Turing machine that accepts
$L_1$ using $L_2$ as its oracle.}
\end{definition}

We now are ready to formally define NP-hardness and NP-completeness.  In the
following, $\leq^p_r$ stands for either many-one or Turing polynomial-time
reductions.

\begin{definition}
{\rm A language $L$ is NP-{\em hard under $\leq^p_r$ reductions\/}
(alternatively, \mbox{$\leq^p_r$-{\em hard}} {\em for\/} NP) if
$$
  \forall L^\prime\in {\rm NP}\ (L^\prime\leq^p_r L).
$$}
\end{definition}

\begin{definition}
{\rm A language $L$ is NP-{\em complete under $\leq^p_r$ reductions\/}
(alternatively, \mbox{$\leq^p_r$-{\em complete for\/}} NP) if $L\in$ NP and
$L$ is NP-hard under $\leq^p_r$ reductions.}
\end{definition}

The following theorem says that unless P=NP, no one will ever be able to 
find a polynomial-time algorithm to solve any NP-hard problem.

\begin{theorem}
If a language $L$ is\/ {\rm NP}-hard under Turing reductions and $L\in$
{\rm P}, then\/ {\rm P=NP}.\\[1pc]
{\rm\bf Corollary} If a language $L$ is\/ {\rm NP}-hard under many-one
reductions and $L\in$ {\rm P}, then\/ {\rm P=NP}.
\end{theorem}

Thus, if someone some day finds a polynomial-time algorithm
for any NP-hard problem, then this algorithm could be used to solve any
NP-complete problem in polynomial time, which would solve many important
open problems (and would also be very useful).  However, the current
wisdom is that this will never be possible.  This means that proving that a
problem is NP-hard is strong evidence that no polynomial-time algorithm
exists for that problem.  For this reason, NP-hard problems are said to be
{\em computationally intractable}.

Once we have a collection of problems that are proven to be NP-hard,
these problems can be used to prove other problems are NP-hard as well via
reductions.  This is the method most used to prove NP-hardness, and it is
precisely how we prove our main result.
This method is formalized by the following theorem:

\begin{theorem}
If $L$ is\/ {\rm NP}-hard under $\leq^p_r$ reductions and
$L \leq^p_r L^\prime$ then $L^\prime$ is\/ {\rm NP}-hard under
$\leq^p_r$ reductions.
\end{theorem}

It should be noted that NP-completeness is usually defined using many-one
reductions, but philosophically, it may make more sense to define
NP-completeness using Turing reductions.  This distinction is important
because definitions based on different reductions are believed not to be
identical; however, since many-one reductions are a special case of Turing
reductions, proving that a language is many-one-hard for NP implies that it
is Turing-hard for NP.


%%% Local Variables: 
%%% mode: latex
%%% TeX-master: "thesis"
%%% End: 
